\chapter{Introduction}

\section{Motivation}

Wine quality assessment traditionally relies on expert sensory evaluation, which can be subjective and inconsistent. With the advancement of analytical chemistry and data science, there's an opportunity to develop objective approaches to wine quality prediction. This study applies statistical methods learned in the Introduction to Data Science course to analyze a Portuguese wine dataset, examining the relationship between chemical composition and quality ratings to uncover patterns that could inform wine production and assessment practices.\\

\section{Problem Definition}

This study aims to explore whether wine quality and characteristics can be predicted from chemical composition data. As students learning data science methods, we want to investigate several practical questions: Can we identify the key chemical differences between red and white wines? Is it possible to predict wine quality from laboratory measurements? How well can we classify wines into different categories using only their chemical properties ?\\

To answer these questions, we will analyze a dataset of Portuguese wines using six different analytical approaches: exploring the data distributions and patterns, testing whether red and white wines differ in alcohol content, building a model to predict quality ratings, creating classifiers to identify good versus bad wines, developing a system to distinguish red from white wines, and reducing the complexity of chemical measurements to identify the most important underlying factors.\\