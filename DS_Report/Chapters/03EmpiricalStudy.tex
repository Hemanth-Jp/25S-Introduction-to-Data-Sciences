\chapter{Own Empirical Study}

\section{Dataset Description}

The analysis utilizes a dataset of Portuguese wines containing 6,497 observations across 13 variables. The dataset includes both red and white wine varieties, with each observation representing a unique wine sample analyzed for chemical composition and rated for quality by expert panels.

\subsection{Variable Descriptions}

\textbf{Chemical Variables:}
\begin{itemize}
	\item \textbf{Fixed Acidity (g/L):} Non-volatile acids that do not evaporate readily
	\item \textbf{Volatile Acidity (g/L):} Acetic acid content, associated with vinegar taste at high levels
	\item \textbf{Citric Acid (g/L):} Adds freshness and flavor in small quantities
	\item \textbf{Residual Sugar (g/L):} Remaining sugar after fermentation completion
	\item \textbf{Chlorides (g/L):} Salt content in wine
	\item \textbf{Free Sulfur Dioxide (mg/L):} Prevents microbial growth and oxidation
	\item \textbf{Total Sulfur Dioxide (mg/L):} Combined free and bound SO$_2$ forms
	\item \textbf{Density (g/mL):} Wine density relative to water
	\item \textbf{pH:} Acidity/basicity measure on 0-14 scale
	\item \textbf{Sulphates (g/L):} Wine additive contributing to SO$_2$ levels
	\item \textbf{Alcohol (\% vol):} Ethanol content by volume
\end{itemize}

\textbf{Target Variables:}
\begin{itemize}
	\item \textbf{Quality:} Expert rating on 0-10 scale (higher = better quality)
	\item \textbf{Variety:} Wine type (red or white)
\end{itemize}