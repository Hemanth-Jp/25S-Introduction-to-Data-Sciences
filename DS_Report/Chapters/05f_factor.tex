%correct version
\section{Factor Analysis}

\subsection{Suitability Assessment}

\begin{lstlisting}[language=R, caption=Factor Analysis Preparation, breaklines=true]
	# Prepare data for factor analysis (chemical and sensory variables only)
	factor_data <- wine_data[, predictor_vars]
	
	# Check for missing values
	cat("Missing values in factor analysis data:\n")
	print(sapply(factor_data, function(x) sum(is.na(x))))
	
	# Correlation matrix assessment
	cat("\n--- Correlation Matrix Suitability ---\n")
	correlation_matrix <- cor(factor_data, use = "complete.obs")
\end{lstlisting}

\begin{lstlisting}[language=R, caption=KMO and Bartlett Tests, breaklines=true]
	# Kaiser-Meyer-Olkin (KMO) Test
	kmo_result <- KMOS(factor_data)
	kmo_overall <- kmo_result$KMO
	msa_values <- kmo_result$MSA
	
	cat("Overall KMO value:", round(kmo_overall, 4), "\n")
	if(kmo_overall >= 0.8) {
		cat("KMO assessment: Excellent for factor analysis\n")
	} else if(kmo_overall >= 0.7) {
		cat("KMO assessment: Good for factor analysis\n")
	} else if(kmo_overall >= 0.6) {
		cat("KMO assessment: Adequate for factor analysis\n")
	} else if(kmo_overall >= 0.5) {
		cat("KMO assessment: Poor but acceptable\n")
	} else {
		cat("KMO assessment: Unacceptable for factor analysis\n")
	}
	
	# Bartlett's Test of Sphericity
	bartlett_result <- cortest.bartlett(correlation_matrix, n = nrow(factor_data))
	cat("\nBartlett's Test of Sphericity: p =", bartlett_result$p.value, "\n")
	
	if(bartlett_result$p.value < 0.05) {
		cat("Bartlett's test: Correlations exist (suitable for factor analysis)\n")
	} else {
		cat("Bartlett's test: No significant correlations (not suitable)\n")
	}
\end{lstlisting}

\textbf{Kaiser-Meyer-Olkin (KMO) Test:}
\begin{itemize}
	\item Overall KMO = [X.XXX]
	\item Assessment: [Excellent/Good/Adequate/Poor/Unacceptable] for factor analysis
\end{itemize}

\textbf{Bartlett's Test of Sphericity:}
\begin{itemize}
	\item p-value < 0.001: Correlations exist, suitable for factor analysis
\end{itemize}

\textbf{Measure of Sampling Adequacy (MSA):}
\textit{[Insert MSA values for individual variables]}

\subsection{Factor Extraction}

\begin{lstlisting}[language=R, caption=Determining Number of Factors, breaklines=true]
	# MSA for individual variables
	cat("\nMeasure of Sampling Adequacy (MSA) for individual variables:\n")
	msa_df <- data.frame(Variable = names(msa_values), 
	MSA = round(msa_values, 4))
	print(msa_df)
	
	# Remove variables with MSA < 0.5 if any
	low_msa_vars <- names(msa_values)[msa_values < 0.5]
	if(length(low_msa_vars) > 0) {
		cat("\nVariables with MSA < 0.5 (consider removal):", low_msa_vars, "\n")
		factor_data_reduced <- factor_data[, !names(factor_data) %in% low_msa_vars]
	} else {
		cat("\nAll variables have acceptable MSA (>= 0.5)\n")
		factor_data_reduced <- factor_data
	}
	
	# Determine number of factors
	cat("\n--- Determining Number of Factors ---\n")
	
	# Eigenvalues
	eigenvalues <- eigen(cor(factor_data_reduced))$values
	cat("Eigenvalues:\n")
	for(i in 1:length(eigenvalues)) {
		cat("Factor", i, ":", round(eigenvalues[i], 4), "\n")
	}
	
	# Kaiser criterion (eigenvalues > 1)
	n_factors_kaiser <- sum(eigenvalues > 1)
	cat("\nKaiser criterion (eigenvalues > 1):", n_factors_kaiser, "factors\n")
	
	# Scree plot
	plot(1:length(eigenvalues), eigenvalues, type = "b",
	main = "Scree Plot", xlab = "Factor Number", ylab = "Eigenvalue")
	abline(h = 1, col = "red", lty = 2)
\end{lstlisting}

\textbf{Eigenvalue Analysis:}
\textit{[Insert eigenvalues and Kaiser criterion results]}

\textbf{Scree Plot Interpretation:}
[Description of scree plot and elbow criterion]

\subsection{Factor Analysis Results}

\begin{lstlisting}[language=R, caption=Factor Analysis Implementation, breaklines=true]
	# Conduct factor analysis
	cat("\n--- Factor Analysis Results ---\n")
	
	# Try different numbers of factors
	for(nf in 1:min(4, n_factors_kaiser)) {
		cat("\n=== Factor Analysis with", nf, "factor(s) ===\n")
		
		fa_result <- principal(factor_data_reduced, nfactors = nf, 
		rotate = "varimax")
		
		# Factor loadings
		cat("Factor Loadings (>0.4 shown):\n")
		print(fa_result, cut = 0.4, sort = TRUE)
		
		# Variance explained
		variance_explained <- fa_result$values[1:nf]
		total_variance <- sum(variance_explained)
		proportion_variance <- total_variance / ncol(factor_data_reduced)
		
		cat("\nVariance Explained:\n")
		cat("Total eigenvalues for", nf, "factors:", 
		round(total_variance, 4), "\n")
		cat("Proportion of variance explained:", 
		round(proportion_variance, 4), "\n")
		cat("Percentage of variance explained:", 
		round(proportion_variance * 100, 2), "%\n")
	}
\end{lstlisting}

\textbf{[X]-Factor Solution:}
\textit{[Insert factor loadings table with >0.4 loadings]}

\textbf{Variance Explained:}
\begin{itemize}
	\item Total variance explained: [XX.X]\%
	\item Interpretation of factors: [Describe what each factor represents]
\end{itemize}

\textbf{Factor Interpretation:}
\begin{itemize}
	\item Factor 1: [Description based on loadings]
	\item Factor 2: [Description based on loadings]
	\item [Continue for additional factors]
\end{itemize}
