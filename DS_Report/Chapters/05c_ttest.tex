\section{Hypothesis Testing}

\subsection{Research Question: Alcohol Content Comparison}

\textbf{Objective}: Determine whether red and white wines differ significantly in alcohol content.

\begin{lstlisting}[language=R, caption=Alcohol Content Analysis by Wine Type, breaklines=true]
	# Separate alcohol content by wine variety
	red_alcohol <- wine_data$alcohol[wine_data$variety == "red"]
	white_alcohol <- wine_data$alcohol[wine_data$variety == "white"]
	
	cat("--- Descriptive Statistics by Wine Type ---\n")
	cat("Red wine alcohol content:\n")
	cat("Mean:", mean(red_alcohol, na.rm = TRUE), "\n")
	cat("SD:", sd(red_alcohol, na.rm = TRUE), "\n")
	cat("N:", length(red_alcohol), "\n\n")
	
	cat("White wine alcohol content:\n")
	cat("Mean:", mean(white_alcohol, na.rm = TRUE), "\n")
	cat("SD:", sd(white_alcohol, na.rm = TRUE), "\n")
	cat("N:", length(white_alcohol), "\n\n")
\end{lstlisting}

\subsection{Assumption Checking}

\begin{lstlisting}[language=R, caption=T-Test Assumption Testing, breaklines=true]
	# Check t-test assumptions
	cat("--- Checking T-Test Assumptions ---\n")
	
	# 1. Normality check using Shapiro-Wilk test
	cat("1. Normality Tests:\n")
	shapiro_red <- shapiro.test(red_alcohol)
	shapiro_white <- shapiro.test(white_alcohol)
	
	cat("Red wine alcohol normality (Shapiro-Wilk): p =", 
	shapiro_red$p.value, "\n")
	cat("White wine alcohol normality (Shapiro-Wilk): p =", 
	shapiro_white$p.value, "\n")
	
	# 2. Equal variances test (Levene's test)
	cat("\n2. Equal Variances Test (Levene):\n")
	levene_test <- leveneTest(alcohol ~ variety, data = wine_data)
	print(levene_test)
	
	# 3. Independence assumption (addressed in interpretation)
	cat("\n3. Independence: Assumed based on random sampling design\n")
\end{lstlisting}

\textbf{Normality Assessment:}
\begin{itemize}
	\item Shapiro-Wilk test for red wines: p = [X.XXX]
	\item Shapiro-Wilk test for white wines: p = [X.XXX]
\end{itemize}

\textbf{Equal Variances Assessment:}
\begin{itemize}
	\item Levene's test: F = [X.XX], p = [X.XXX]
\end{itemize}

\textbf{Independence}: Assumed based on sampling methodology

\subsection{T-Test Results}

\begin{lstlisting}[language=R, caption=Two-Sample T-Test, breaklines=true]
	# Conduct t-test
	cat("\n--- Two-Sample T-Test Results ---\n")
	
	# Use Welch t-test (unequal variances) as default
	t_test_result <- t.test(red_alcohol, white_alcohol, var.equal = FALSE)
	print(t_test_result)
	
	# Effect size (Cohen's d)
	pooled_sd <- sqrt(((length(red_alcohol)-1)*var(red_alcohol) + 
	(length(white_alcohol)-1)*var(white_alcohol)) / 
	(length(red_alcohol) + length(white_alcohol) - 2))
	cohens_d <- (mean(red_alcohol) - mean(white_alcohol)) / pooled_sd
	
	cat("\nEffect Size (Cohen's d):", cohens_d, "\n")
\end{lstlisting}

\textit{[Insert t-test output from R]}

\textbf{Statistical Decision}: 
[Based on p-value, state whether to reject or fail to reject $H_0$]

\textbf{Effect Size}: Cohen's d = [X.XX], indicating [small/medium/large] effect size.

\textbf{Interpretation}: [Provide practical interpretation of the results]