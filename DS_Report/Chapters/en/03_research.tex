\chapter{State of Research}

\section{Wine Quality Prediction Studies}

Recent literature demonstrates significant interest in developing predictive models for wine quality assessment. Cortez et al. \cite{cortez2009} pioneered the use of machine learning techniques on Portuguese wine data, achieving moderate success in predicting quality ratings from physicochemical properties. Their work established the foundation for subsequent research in this domain.

Gupta \cite{gupta2018} applied various machine learning algorithms to wine quality prediction, comparing the performance of random forests, support vector machines, and neural networks. The study found that ensemble methods generally outperformed individual algorithms, with chemical acidity and alcohol content being among the most important predictive factors.

More recent research by Kumar et al. \cite{kumar2020} explored deep learning approaches for wine quality assessment, achieving improved prediction accuracy compared to traditional methods. However, the authors noted that the interpretability of deep learning models remains a challenge for practical applications in the wine industry.

\section{Chemical Analysis in Viticulture}

The relationship between wine chemistry and quality has been extensively studied in enological research. Ribéreau-Gayon et al. \cite{ribereau2017} provide a comprehensive overview of wine chemistry, highlighting the importance of compounds such as phenolics, organic acids, and volatile compounds in determining wine quality and character.

Specific chemical parameters have been identified as quality indicators. Volatile acidity, primarily acetic acid, is generally associated with wine defects when present at elevated levels \cite{jackson2020}. Conversely, appropriate levels of fixed acidity contribute to wine structure and stability.

Sulfur dioxide management represents a critical aspect of wine production, with both free and total sulfur dioxide levels requiring careful monitoring to prevent oxidation while avoiding excessive sulfur character \cite{waterhouse2016}.

\section{Classification Techniques in Food Industry}

Classification methods have found widespread application in food quality assessment and authenticity testing. Downey et al. \cite{downey2006} demonstrated the use of spectroscopic techniques combined with chemometric analysis for wine origin classification, achieving high accuracy in distinguishing wines from different geographical regions.

Logistic regression has proven particularly effective for binary classification tasks in food science, such as distinguishing between acceptable and unacceptable products based on quality parameters \cite{granato2018}. The method's interpretability makes it valuable for regulatory applications where decision reasoning must be transparent.

Factor analysis and principal component analysis have been widely used to understand the underlying structure of complex food composition data \cite{jolliffe2016}. These techniques help identify the most important chemical factors that contribute to food quality and characteristics.