\appendix

\chapter{AI Tool Usage Documentation}

\begin{table}[H]
\centering
\caption{AI Tool Usage Documentation}
\begin{tabular}{|p{3cm}|p{4cm}|p{4cm}|p{4cm}|}
\hline
\textbf{Used Tool} & \textbf{Type of Use} & \textbf{Affected Parts of Work} & \textbf{Remarks} \\
\hline
Claude (Anthropic) & R code structure and debugging assistance & Section 5, R Script & Code templates and debugging help. Prompts included in Appendix C. \\
\hline
Claude (Anthropic) & Statistical method consultation & Section 2, Section 5 & Verification of statistical procedures. Original interpretations by author. \\
\hline
DeepL Translator & Translation assistance & Literature review & Minor assistance with German sources \\
\hline
\end{tabular}
\end{table}

\chapter{R Script Output}

\textit{[Include all R script outputs, including tables, statistical test results, and any relevant screenshots]}

\chapter{AI Conversation Prompts}

\textit{[Include relevant prompts used with AI tools, showing transparency in assistance received]}

\subsection{Prompt 1: R Script Development}
``Please help me create an R script structure for wine dataset analysis including exploratory data analysis, t-tests, linear regression, logistic regression, and factor analysis...''

\subsection{Prompt 2: Statistical Interpretation}
``How should I interpret VIF values in multiple regression, and what constitutes multicollinearity?''

\textit{[Continue with other relevant prompts]}

\chapter{Complete R Script}

\begin{lstlisting}[language=R, caption=Complete Wine Analysis R Script, breaklines=true]
# =============================================================================
# Wine Dataset Analysis - Introduction to Data Science Term Paper
# Prof. Dr. Joachim Schwarz - Emden/Leer University of Applied Sciences
# 
# AI Usage Documentation:
# - Claude (Anthropic) was used for R code structure and debugging assistance
# - Prompts included in appendix of term paper
# - Statistical interpretations are original author analysis
# =============================================================================

# Clear workspace and set working directory
rm(list = ls())

# Required Libraries
# Install packages if not already installed
required_packages <- c("psych", "car", "lmtest", "corrplot", "REdaS", 
                      "pROC", "ROCR", "caret", "ggplot2", "dplyr")

for(pkg in required_packages) {
  if(!require(pkg, character.only = TRUE)) {
    install.packages(pkg)
    library(pkg, character.only = TRUE)
  }
}

# =============================================================================
# DATA IMPORT AND INITIAL SETUP
# =============================================================================

# Read the wine dataset
wine_data <- read.csv("wine copy.csv", stringsAsFactors = FALSE)

# Remove the index column (X) as specified in assignment
wine_data <- wine_data[, -1]

# Display basic information about the dataset
cat("=== WINE DATASET OVERVIEW ===\n")
cat("Dataset dimensions:", dim(wine_data), "\n")
cat("Number of observations:", nrow(wine_data), "\n")
cat("Number of variables:", ncol(wine_data), "\n\n")

# Display structure
str(wine_data)

# Convert variety to factor
wine_data$variety <- as.factor(wine_data$variety)

# =============================================================================
# TASK 1: EXPLORATORY DATA ANALYSIS (EDA)
# =============================================================================

cat("\n=== TASK 1: EXPLORATORY DATA ANALYSIS ===\n")

# a) Summary statistics for all metric variables
cat("\n--- Summary Statistics for All Metric Variables ---\n")

# Select numeric variables (exclude variety)
numeric_vars <- wine_data[, sapply(wine_data, is.numeric)]

# Comprehensive summary statistics
summary_stats <- describe(numeric_vars)
print(summary_stats)

# Check for missing values
cat("\n--- Missing Values Analysis ---\n")
missing_values <- sapply(wine_data, function(x) sum(is.na(x)))
print(missing_values)

if(sum(missing_values) == 0) {
  cat("No missing values found in the dataset.\n")
} else {
  cat("Missing values detected. See above for details.\n")
}

# Frequency distributions for categorical variables
cat("\n--- Frequency Distributions for Categorical Variables ---\n")
cat("Wine Variety Distribution:\n")
variety_table <- table(wine_data$variety)
print(variety_table)
print(prop.table(variety_table))

cat("\nQuality Distribution:\n")
quality_table <- table(wine_data$quality)
print(quality_table)
print(prop.table(quality_table))

# b) Create visualizations and calculate skewness
cat("\n--- Generating Graphics and Skewness Analysis ---\n")

# Set up plotting parameters
par(mfrow = c(2, 2))

# Loop through numeric variables for histograms and skewness
skewness_results <- data.frame(
  Variable = names(numeric_vars),
  Skewness = numeric(ncol(numeric_vars)),
  Interpretation = character(ncol(numeric_vars)),
  stringsAsFactors = FALSE
)

for(i in 1:ncol(numeric_vars)) {
  var_name <- names(numeric_vars)[i]
  var_data <- numeric_vars[, i]
  
  # Calculate skewness
  skew_val <- skew(var_data, na.rm = TRUE)
  skewness_results$Skewness[i] <- skew_val
  
  # Interpret skewness
  if(abs(skew_val) < 0.5) {
    skewness_results$Interpretation[i] <- "Approximately symmetric"
  } else if(skew_val >= 0.5) {
    skewness_results$Interpretation[i] <- "Right-skewed"
  } else {
    skewness_results$Interpretation[i] <- "Left-skewed"
  }
  
  # Create histogram
  hist(var_data, 
       main = paste("Histogram of", var_name),
       xlab = var_name,
       col = "lightblue",
       border = "black")
}

# Reset plotting parameters
par(mfrow = c(1, 1))

cat("\nSkewness Analysis Results:\n")
print(skewness_results)

# Boxplots for outlier detection
cat("\n--- Outlier Detection using Boxplots ---\n")

# Create boxplots for key variables
par(mfrow = c(2, 3))
key_vars <- c("fixed.acidity", "volatile.acidity", "alcohol", "quality", "pH", "sulphates")

for(var in key_vars) {
  if(var %in% names(wine_data)) {
    boxplot(wine_data[[var]], 
            main = paste("Boxplot of", var),
            ylab = var,
            col = "lightgreen")
  }
}

par(mfrow = c(1, 1))

# =============================================================================
# TASK 2: HYPOTHESIS TESTING - ALCOHOL CONTENT BY WINE TYPE
# =============================================================================

cat("\n=== TASK 2: HYPOTHESIS TESTING ===\n")

# Research question: Does alcohol content differ significantly between red and white wines?

# Separate alcohol content by wine variety
red_alcohol <- wine_data$alcohol[wine_data$variety == "red"]
white_alcohol <- wine_data$alcohol[wine_data$variety == "white"]

cat("--- Descriptive Statistics by Wine Type ---\n")
cat("Red wine alcohol content:\n")
cat("Mean:", mean(red_alcohol, na.rm = TRUE), "\n")
cat("SD:", sd(red_alcohol, na.rm = TRUE), "\n")
cat("N:", length(red_alcohol), "\n\n")

cat("White wine alcohol content:\n")
cat("Mean:", mean(white_alcohol, na.rm = TRUE), "\n")
cat("SD:", sd(white_alcohol, na.rm = TRUE), "\n")
cat("N:", length(white_alcohol), "\n\n")

# Check t-test assumptions
cat("--- Checking T-Test Assumptions ---\n")

# 1. Normality check using Shapiro-Wilk test
cat("1. Normality Tests:\n")
shapiro_red <- shapiro.test(red_alcohol)
shapiro_white <- shapiro.test(white_alcohol)

cat("Red wine alcohol normality (Shapiro-Wilk): p =", shapiro_red$p.value, "\n")
cat("White wine alcohol normality (Shapiro-Wilk): p =", shapiro_white$p.value, "\n")

# 2. Equal variances test (Levene's test)
cat("\n2. Equal Variances Test (Levene):\n")
levene_test <- leveneTest(alcohol ~ variety, data = wine_data)
print(levene_test)

# 3. Independence assumption (addressed in interpretation)
cat("\n3. Independence: Assumed based on random sampling design\n")

# Conduct t-test
cat("\n--- Two-Sample T-Test Results ---\n")

# Use Welch t-test (unequal variances) as default
t_test_result <- t.test(red_alcohol, white_alcohol, var.equal = FALSE)
print(t_test_result)

# Effect size (Cohen's d)
pooled_sd <- sqrt(((length(red_alcohol)-1)*var(red_alcohol) + 
                  (length(white_alcohol)-1)*var(white_alcohol)) / 
                  (length(red_alcohol) + length(white_alcohol) - 2))
cohens_d <- (mean(red_alcohol) - mean(white_alcohol)) / pooled_sd

cat("\nEffect Size (Cohen's d):", cohens_d, "\n")

# Interpretation
cat("\n--- Interpretation ---\n")
if(t_test_result$p.value < 0.05) {
  cat("Result: Significant difference in alcohol content between wine types (p < 0.05)\n")
} else {
  cat("Result: No significant difference in alcohol content between wine types (p ≥ 0.05)\n")
}

# =============================================================================
# TASK 3: LINEAR REGRESSION (RED WINES ONLY)
# =============================================================================

cat("\n=== TASK 3: LINEAR REGRESSION ANALYSIS (RED WINES ONLY) ===\n")

# Filter for red wines only
red_wines <- wine_data[wine_data$variety == "red", ]
cat("Number of red wine observations:", nrow(red_wines), "\n")

# Select chemical and sensory variables (exclude variety and quality initially)
predictor_vars <- c("fixed.acidity", "volatile.acidity", "citric.acid", 
                   "residual.sugar", "chlorides", "free.sulfur.dioxide", 
                   "total.sulfur.dioxide", "density", "pH", "sulphates", "alcohol")

# Create regression model
cat("\n--- Multiple Linear Regression Model ---\n")
regression_formula <- as.formula(paste("quality ~", paste(predictor_vars, collapse = " + ")))
wine_lm <- lm(regression_formula, data = red_wines)

# Display regression results
summary(wine_lm)

# Store key regression statistics
r_squared <- summary(wine_lm)$r.squared
adj_r_squared <- summary(wine_lm)$adj.r.squared
f_statistic <- summary(wine_lm)$fstatistic

cat("\nKey Model Statistics:\n")
cat("R-squared:", round(r_squared, 4), "\n")
cat("Adjusted R-squared:", round(adj_r_squared, 4), "\n")
cat("F-statistic:", round(f_statistic[1], 4), "\n")

# =============================================================================
# REGRESSION DIAGNOSTICS
# =============================================================================

cat("\n--- Regression Diagnostics ---\n")

# 1. Linearity Check - Residuals vs Fitted Plot
cat("1. Linearity Assessment:\n")
plot(fitted(wine_lm), resid(wine_lm),
     main = "Residuals vs Fitted Values",
     xlab = "Fitted Values", ylab = "Residuals")
abline(h = 0, col = "red", lty = 2)

# RESET test for linearity
reset_test <- resettest(wine_lm, power = 2:3, type = "fitted")
cat("RESET Test for Linearity: p =", reset_test$p.value, "\n")

# 2. Normality of Residuals
cat("\n2. Normality of Residuals:\n")
shapiro_resid <- shapiro.test(resid(wine_lm))
cat("Shapiro-Wilk test on residuals: p =", shapiro_resid$p.value, "\n")

# Q-Q plot
qqnorm(resid(wine_lm))
qqline(resid(wine_lm), col = "red")

# 3. Homoscedasticity (Equal Variances)
cat("\n3. Homoscedasticity Tests:\n")
bp_test <- bptest(wine_lm)
cat("Breusch-Pagan test: p =", bp_test$p.value, "\n")

# 4. Multicollinearity Check
cat("\n4. Multicollinearity Assessment:\n")
vif_values <- vif(wine_lm)
print(vif_values)

cat("\nVIF Interpretation (>10 indicates multicollinearity):\n")
high_vif <- vif_values[vif_values > 10]
if(length(high_vif) > 0) {
  cat("High VIF variables:", names(high_vif), "\n")
} else {
  cat("No serious multicollinearity detected (all VIF < 10)\n")
}

# 5. Autocorrelation Check
cat("\n5. Autocorrelation Test:\n")