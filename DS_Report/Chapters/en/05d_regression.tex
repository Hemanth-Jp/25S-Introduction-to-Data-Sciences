\section{Linear Regression Analysis}

\subsection{Model Specification}

The linear regression model predicts wine quality using all available chemical and sensory variables for red wines only:

\begin{equation}
	\text{Quality} = \beta_0 + \beta_1(\text{Fixed Acidity}) + \beta_2(\text{Volatile Acidity}) + \ldots + \beta_{11}(\text{Alcohol}) + \varepsilon
\end{equation}

\begin{lstlisting}[language=R, caption=Linear Regression Model Setup]
	# Filter for red wines only
	red_wines <- wine_data[wine_data$variety == "red", ]
	cat("Number of red wine observations:", nrow(red_wines), "\n")
	
	# Select chemical and sensory variables
	predictor_vars <- c("fixed.acidity", "volatile.acidity", "citric.acid", 
	"residual.sugar", "chlorides", "free.sulfur.dioxide", 
	"total.sulfur.dioxide", "density", "pH", "sulphates", "alcohol")
	
	# Create regression model
	cat("\n--- Multiple Linear Regression Model ---\n")
	regression_formula <- as.formula(paste("quality ~", paste(predictor_vars, collapse = " + ")))
	wine_lm <- lm(regression_formula, data = red_wines)
	
	# Display regression results
	summary(wine_lm)
\end{lstlisting}

\subsection{Regression Results}

\textit{[Insert regression summary output]}

\textbf{Model Fit Statistics:}
\begin{itemize}
	\item R² = [X.XXX]: [X]\% of variance in quality explained
	\item Adjusted R² = [X.XXX]: Accounts for number of predictors
	\item F-statistic = [X.XX], p < 0.001: Model is statistically significant
\end{itemize}

\textbf{Significant Predictors} ($\alpha = 0.05$):
\textit{[List significant variables with coefficients and interpretations]}

\subsection{Regression Diagnostics}

\begin{lstlisting}[language=R, caption=Regression Diagnostics]
	cat("\n--- Regression Diagnostics ---\n")
	
	# 1. Linearity Check - Residuals vs Fitted Plot
	cat("1. Linearity Assessment:\n")
	plot(fitted(wine_lm), resid(wine_lm),
	main = "Residuals vs Fitted Values",
	xlab = "Fitted Values", ylab = "Residuals")
	abline(h = 0, col = "red", lty = 2)
	
	# RESET test for linearity
	reset_test <- resettest(wine_lm, power = 2:3, type = "fitted")
	cat("RESET Test for Linearity: p =", reset_test$p.value, "\n")
	
	# 2. Normality of Residuals
	cat("\n2. Normality of Residuals:\n")
	shapiro_resid <- shapiro.test(resid(wine_lm))
	cat("Shapiro-Wilk test on residuals: p =", shapiro_resid$p.value, "\n")
	
	# 3. Homoscedasticity
	cat("\n3. Homoscedasticity Tests:\n")
	bp_test <- bptest(wine_lm)
	cat("Breusch-Pagan test: p =", bp_test$p.value, "\n")
	
	# 4. Multicollinearity Check
	cat("\n4. Multicollinearity Assessment:\n")
	vif_values <- vif(wine_lm)
	print(vif_values)
	
	# 5. Autocorrelation Check
	cat("\n5. Autocorrelation Test:\n")
	dw_test <- dwtest(wine_lm)
	cat("Durbin-Watson test: p =", dw_test$p.value, "\n")
\end{lstlisting}

\textbf{Linearity Assessment:}
\begin{itemize}
	\item RESET test: p = [X.XXX]
	\item Residuals vs. Fitted plot interpretation: [Description]
\end{itemize}

\textbf{Normality of Residuals:}
\begin{itemize}
	\item Shapiro-Wilk test on residuals: p = [X.XXX]
	\item Q-Q plot assessment: [Description]
\end{itemize}

\textbf{Homoscedasticity:}
\begin{itemize}
	\item Breusch-Pagan test: p = [X.XXX]
\end{itemize}

\textbf{Multicollinearity:}
\begin{itemize}
	\item VIF values: [List any values > 10 and interpretation]
\end{itemize}

\textbf{Autocorrelation:}
\begin{itemize}
	\item Durbin-Watson test: p = [X.XXX]
\end{itemize}

\textbf{Assumption Summary}: [Overall assessment of regression assumptions and any violations]