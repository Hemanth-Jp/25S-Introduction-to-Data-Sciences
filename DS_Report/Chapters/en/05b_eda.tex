\section{Exploratory Data Analysis}

\subsection{Summary Statistics}

\begin{lstlisting}[language=R, caption=Summary Statistics for Metric Variables]
	# Select numeric variables (exclude variety)
	numeric_vars <- wine_data[, sapply(wine_data, is.numeric)]
	
	# Comprehensive summary statistics
	summary_stats <- describe(numeric_vars)
	print(summary_stats)
\end{lstlisting}

\textit{[Insert R output from describe() function showing mean, standard deviation, median, quartiles, min, max for all numeric variables]}

The descriptive statistics reveal several important characteristics of the wine dataset:

\textbf{Central Tendency}: Most chemical variables show reasonable distributions around their means, with quality ratings averaging approximately [X.X] on the 10-point scale.

\textbf{Variability}: Standard deviations indicate moderate variability in most chemical parameters, with residual sugar showing the highest coefficient of variation, suggesting diverse wine styles in the dataset.

\textbf{Missing Values}: Analysis confirmed no missing values in the dataset, ensuring complete case analysis for all statistical procedures.

\begin{lstlisting}[language=R, caption=Missing Values Analysis]
	# Check for missing values
	cat("\n--- Missing Values Analysis ---\n")
	missing_values <- sapply(wine_data, function(x) sum(is.na(x)))
	print(missing_values)
	
	if(sum(missing_values) == 0) {
		cat("No missing values found in the dataset.\n")
	} else {
		cat("Missing values detected. See above for details.\n")
	}
\end{lstlisting}

\subsection{Distribution Analysis and Skewness}

\begin{lstlisting}[language=R, caption=Skewness Analysis and Visualization]
	# Loop through numeric variables for histograms and skewness
	skewness_results <- data.frame(
	Variable = names(numeric_vars),
	Skewness = numeric(ncol(numeric_vars)),
	Interpretation = character(ncol(numeric_vars)),
	stringsAsFactors = FALSE
	)
	
	for(i in 1:ncol(numeric_vars)) {
		var_name <- names(numeric_vars)[i]
		var_data <- numeric_vars[, i]
		
		# Calculate skewness
		skew_val <- skew(var_data, na.rm = TRUE)
		skewness_results$Skewness[i] <- skew_val
		
		# Interpret skewness
		if(abs(skew_val) < 0.5) {
			skewness_results$Interpretation[i] <- "Approximately symmetric"
		} else if(skew_val >= 0.5) {
			skewness_results$Interpretation[i] <- "Right-skewed"
		} else {
			skewness_results$Interpretation[i] <- "Left-skewed"
		}
		
		# Create histogram
		hist(var_data, 
		main = paste("Histogram of", var_name),
		xlab = var_name,
		col = "lightblue",
		border = "black")
	}
\end{lstlisting}

\textit{[Insert skewness analysis table from R output]}

Skewness analysis reveals important distributional characteristics:

\begin{itemize}
	\item \textbf{Right-skewed variables} (skewness > 0.5): [List variables] suggest the presence of wines with elevated levels of these compounds
	\item \textbf{Left-skewed variables} (skewness < -0.5): [List variables] indicate few wines with very low levels
	\item \textbf{Approximately symmetric variables} (|skewness| < 0.5): [List variables] follow roughly normal distributions
\end{itemize}

\subsection{Outlier Detection}

\begin{lstlisting}[language=R, caption=Outlier Detection using Boxplots]
	# Create boxplots for key variables
	par(mfrow = c(2, 3))
	key_vars <- c("fixed.acidity", "volatile.acidity", "alcohol", 
	"quality", "pH", "sulphates")
	
	for(var in key_vars) {
		if(var %in% names(wine_data)) {
			boxplot(wine_data[[var]], 
			main = paste("Boxplot of", var),
			ylab = var,
			col = "lightgreen")
		}
	}
	par(mfrow = c(1, 1))
\end{lstlisting}

Boxplot analysis identified potential outliers in several variables:

\textit{[Insert interpretation of boxplot results]}

These outliers may represent wines with exceptional characteristics or measurement errors, but were retained in the analysis as they may contain valuable information about wine diversity.

\subsection{Categorical Variable Distributions}

\begin{lstlisting}[language=R, caption=Frequency Distributions for Categorical Variables]
	# Frequency distributions for categorical variables
	cat("\n--- Frequency Distributions for Categorical Variables ---\n")
	cat("Wine Variety Distribution:\n")
	variety_table <- table(wine_data$variety)
	print(variety_table)
	print(prop.table(variety_table))
	
	cat("\nQuality Distribution:\n")
	quality_table <- table(wine_data$quality)
	print(quality_table)
	print(prop.table(quality_table))
\end{lstlisting}

\textbf{Wine Variety Distribution:}
\begin{itemize}
	\item Red wines: [X] observations ([X]\%)
	\item White wines: [X] observations ([X]\%)
\end{itemize}

\textbf{Quality Rating Distribution:}
\textit{[Insert quality distribution table and interpretation]}