\chapter{Data Description and Methodology}

\section{Dataset Overview}
The dataset contains comprehensive information on 6,497 Portuguese wines with 14 variables including chemical properties and quality ratings. The distribution shows 1,599 red wines (24.6\%) and 4,898 white wines (75.4\%), providing adequate sample sizes for comparative analysis.

\section{Variables Description}
\textbf{Chemical Properties:} Fixed acidity (7.22 $\pm$ 1.30 g/l), volatile acidity (0.34 $\pm$ 0.17 g/l), citric acid (0.32 $\pm$ 0.15 g/l), residual sugar (5.44 $\pm$ 4.76 g/l), chlorides (0.056 $\pm$ 0.035 g/l), sulfur dioxide levels, density (0.995 $\pm$ 0.003 g/ml), pH (3.22 $\pm$ 0.16), sulphates (0.53 $\pm$ 0.15 g/l), and alcohol content (10.49 $\pm$ 1.19\% vol).

\textbf{Quality Measures:} Quality scores range from 3 to 9 (mean = 5.82 $\pm$ 0.87) with variety classification (red/white).

\section{Statistical Methods Applied}
Following the CRISP-DM methodology, we applied comprehensive statistical techniques including two-sample t-tests with assumption verification, multiple linear regression with diagnostic testing, logistic regression for binary classification, and factor analysis using varimax rotation.